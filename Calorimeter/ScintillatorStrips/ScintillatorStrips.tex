\section{Scintillator Strips}
Most recent update: 2015-04-28 \\
Contact person: Tohru Takeshita (email: tohru@shinshu-u.ac.jp)
\subsection{Introduction}

The CALICE scintillator strip-based ECAL (ScECAL) uses a scintillator  strip
structure to deliver the granularity and resolution required of an ILC detector.
Each strip is individually read out by a Multi Pixel Photon Counter (MPPC, a
silicon photon detector produced by Hamamatsu Photonics K\.K\.~\cite{Gomi:2007zz}). Although
plastic scintillators have been widely used in calorimeters, this is the first
time that a highly granular calorimeter has been made using scintillator strips.
Such an ECAL has a smaller cost than alternative technologies using silicon
sensors (e.g.~\cite{1748-0221-3-08-P08001}). The MPPC has promising properties for the ScECAL: a small
size (active area of $1\times\SI{1}{mm^2}$ in a package of $2.4 \times 1.9 \times \SI{0.85}{mm^3}$), excellent
photon counting ability, low cost and low operation voltage ($\SI{70}{V}$), with
disadvantages of temperature-dependent gain, saturation at high light levels,
and the dark noise rate. The use of tungsten absorber material
minimises the Moliere radius of the calorimeter, an important aspect for the
effective separation of particle showers required by PFA reconstruction. The
chosen strip geometry allows a reduction in the number of readout channels,
while maintaining an effective granularity given by the strip width, by the use
of appropriate reconstruction algorithms. One such algorithm, know as the Strip
Splitting Algorithm~\cite{Kotera2015}, has been developed and demonstrated to perform well in
jets expected at ILC.

\subsection{Recent Milestones}
\begin{itemize}
	\item introducing a new scintillation light readout scheme, with different scintillator strip shape by having better homogeneity
	\item photo-sensor of increased number of pixels in $\SI{1}{mm}\times\SI{1}{mm}$, this leads larger dynamic range for the calorimeter
	\item more experience on the FE read out board and ASICs
\end{itemize}
They are not published yet, instead some proceedings

\subsection{Engineering Challenges}
\begin{itemize}
	\item wrapping the scintillator strip and align them on the FE read out layer automatically
	\item mass test facility for the read out layer
\end{itemize}

\subsection{Future Plans}
\begin{itemize}
	\item deciding on the scintillator layer: shape of scintillator strip, how to read out scintillation light, the location of  photo-sensor, size and shape of photo-sensor and mass production scheme
	\item developing photo-sensor with Hamamatsu photonics company, to have lager dynamic range and mass test scheme
	\item establish a detector fabrication plan
\end{itemize}

\section{Motivation and Constraints for Vertex Detectors at Linear Colliders}

The reconstruction of displaced decays has been an important part of particle physics programs since the days of bubble chambers and the discovery of ``V'' particles. This is still true in today's high-energy collider experiments. If long-lived particles, such as B or D mesons, or tau leptons, decay to at least one charged track in the detector, they can in principle be resolved from the interactions of the primary collision. Similarly, the possibility to distinguish several primary interaction points significantly improves the reconstruction of events. To this end, modern vertex detectors use silicon sensors with small pixels, assembled in support structures with as little material as possible to reduce multiple scattering. A driver for Linear Collider vertex detector R\&D has been low power operation to reduce or completely avoid the need for active cooling.

The reconstruction of displaced vertices depends crucially on the resolution of the perigee, or \emph{impact parameter} of a helical track from the interaction point, which can be parameterized as $\sigma_\text{ip} = \left(\frac{\alpha}{p^{3/2}\sin\theta}\right).$
The goal for the impact parameter resolution in an ILC experiment is $\approx\SI{3}{\micro\meter}$. This puts stringent requirements on the pixel size and the distance of the innermost layer from the interaction point.

Machine-induced background processes at the ILC and CLIC include electron--positron pairs produced in beam--beam interactions. During collisions, vertex detectors accumulate hits from these processes, in addition to those from the primary physics events. In addition to small feature size and low power, it is the reduction of these backgrounds that drives the R\&D for linear collider vertex detectors. Because of the close proximity to the beams, their design is heavily influenced by the beam structure of the accelerator.

In addition to the precise determination of decay vertices, the CLICdp detector and the SiD concept have integrated vertex detectors into their all-silicon tracking systems. This puts additional emphasis on high reliability and high hit detection efficiency.

Highly performing vertex detectors are essential for the success of the physics program at the Linear Collider. Discovery channels for a large class of new physics models involve third-generation fermions: b quarks that form long-lived mesons, top quarks that decay predominantly to a b quark and a W boson, and long-lived tau leptons.

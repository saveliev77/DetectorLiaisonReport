\section{LCSim}

\subsection{Introduction}
The lcsim physics and detector response simulation and event reconstruction
toolkit provides a suite of
software programs to allow studies of multiple detector
designs for the ILC. These tools
include the Geant4-based detector response simulation program (slic), and the Java-based reconstruction and analysis tools (org.lcsim)~\cite{lcsimWebpage}.

\subsection{Recent Milestones}
{\color{red} Mile stones since the DBD}

\subsection{Engineering Challenges}
{\color{red} What are the challenges for using this when the ILC becomes real?}

\subsection{Future Plans}
The core functionality is being kept current by upgrading to the latest versions
of Geant4, etc. Due to lack of funding, the project is currently primarily
responding to user requests for additional functionality.

\subsection{Applications Outside of Linear Colliders}
The flexibility and power of this simulation package make it not only useful for
the application domain for which it was developed (viz. HEP collider detector
physics), but also for other physics experiments, and could very easily be
applied to other disciplines, e.g. biomedical or aerospace, to efficiently use
the full power of the Geant4 toolkit to simulate the interaction of particles
with fields and matter. The Heavy Photon Search experiment at Thomas
Jefferson National Laboratory has adopted slic as its detector response
simulation package and the org.lcsim toolkit for its event reconstruction needs.
Physics and detector studies for CLIC and the Muon Collider have also
used both slic and the org.lcsim software. The software could be easily used for
physics and detector studies at detectors at future circular colliders.

\section{Resistive charge-division on thinned micro-strips sensors with low signal amplification}
Most recent update: 2015-11-15\\
Contact person: Ivan Vila Alvarez (email: ivan.vila@csic.es)
\subsection{Introduction and Motivation}

In the context of the ILC, the relatively low occupancy environment and the power pulsing operation of the front-end electronics provide an opportunity for the implementation of ultra-lightweight silicon-based tracking systems where the dominant contribution to the material budget in the fiducial volume comes from the sensors. Reducing the material budget has a major impact on the hit position resolution and hence the momentum resolution of the tracker system; therefore, we have pursued during the last three years an RD program for the development of very thin micro-strips sensors able to provide two dimensional information of the hit position.

The ultimate goal of this R\&D is the development of a micro-strip sensor which combines signal amplification --- allowing the thinning of the sensor’s substrate --- and resistive electrodes --- allowing the implementation of the charge-division method for the determination of the hit position along the strip direction. In a first phase, we are aiming to demonstrate the feasibility of each of the above mentioned features independently and, in a second phase, to integrate both technological solutions into the same micro-strip sensor; the thinning of the sensor will done using the anisotropic wet etching (TMAH process) used for the DEPFET fabrication \ref{sec:DEPFET}.

\subsection{Recent Developments and Milestones}

The use of the charge-division method in long micro-strip sensors, with a length of several tens of centimeters, was proposed as a possible tracking technology for the International Linear Collider detector concepts a few years ago~\cite{Carman:2011zz}. More recently, we have demonstrated \cite{2012JInst...7.2005B,Bassignana2013186,Curras:ICHEP2014} the feasibility of the charge division concept on fully fledged micro-strip sensor with resistive electrodes made of poly-cristalline silicon achieving a spatial resolution along the strip direction of about 7\% the strip length. One of the limitations of this technology is the attenuation of the signal along the resistive electrode; additionally, the position resolution along the strip is proportional to the signal--to--noise ratio. Therefore, to maintain or even increase the SNR without increasing the sensor substrate thickness we proposed the integration of signal amplification structures in the sensor itself. The Low Gain Avalanche Detector (LGAD) technology appears as a well suited technique for achieving the signal amplification.
LGAD devices engineered as reach-trough avalanche detectors with a moderate gain where initially proposed and developed for timing application [Pellegrini 2014], the moderate signal amplification ensured that a relatively standard front-end readout electronics could be employed. As a spin-off of this original aim, we introduced the i-LGAD micro-strip concept for tracking, a LGAD device implemented in a p-type substrate where the ohmic electrode is strip-wise segmented; this design favors the uniform signal amplification over the sensors active volume overcoming the non-uniform gain in LGAD micro-strips sensors with a strip-wise segmented amplification layer that we recently characterized~\cite{Pellegrini:Hiroshima2015,Vila:LCWS2015}.
The former R\&D line is complemented with the development of a dedicated ASIC using a \SI{180}{nm} AMS fabrication process which integrates a charge amplifier with long shaping time and time stamping functionalities; finally, we completed the study and testing of several pulsed power system topologies based on super-capacitors.

\subsection{Engineering Challenges}
Concerning the component aspects, the main challenges are to complete to proof-of-concept of the thinned micro-strips with charge amplification and resistive charge-division in a implementation suitable for the LC tracking needs, namely:  proof the i-LGAD concept, integrating amplification and charge division, thinning of sensors substrate, large area sensors, manufacturing long ladder by daisy chaining of the sensors. Concerning the read out ASIC, the main challenge will be the design of the front-end with the required functionalities while keeping the power dissipation low enough.
System wise, the main challenge is the design of an air-based cooling system and its integration on the CFRP supporting structure such that the material budget of the system remains acceptable from the point of view of it tracking performance.

\subsection{Future detector R\&D}
During the next two years we will focus our activities on the testing of the i-LGAD devices and, if the results were positive, the integration of the low gain amplification and manufacturing of large area sensors ($\SI{100}{cm^2}$). Concerning the front-end electronics, the main goal will be to complete a few channel demonstrator integrating the long shaping time amplifier and power pulsing. A real scale thermo-mechanical mockup is of the FTD sub-detector at ILD is currently under construction to assess different air forced cooling options.
